\documentclass[11pt,a4paper]{article}
\usepackage[left=2.5cm,right=2.5cm,top=3cm,bottom=3.5cm]{geometry}
\usepackage[ngerman]{babel}
\usepackage[utf8]{inputenc}
\usepackage{graphicx}
\usepackage{amsfonts} 
\usepackage{svg}
\usepackage{amsmath}
\begin{document}
 
 \begin{center}
  {\scshape\LARGE Grundpraktikum I \par}
  \vspace{1cm}
  {\scshape\Large Versuchsprotokoll\par}
  \vspace{1.5cm}
  {\huge\bfseries Isentropenindex\par}
  \vspace{2cm}
     {\large \itshape{Clemens Schumann, Tassilo Scheffler}\/ \par}
  \vspace{0.5cm}
  {clemensrubenschumann@googlemail.com, \\ tassilo@glief.de}
  \vfill
  betreut von\par
  \textsc{Larissa Melischek}
  \vfill
  {\Large 10.03.2018}
 
 \end{center}
 
 \thispagestyle{empty}
 
 \newpage
 \setcounter{page}{1}
 \tableofcontents
 \newpage
 \section{\underline{Physikalische Grundlagen}}
 \subsection{Innere Energie und W\"armekapazit\"at}
 In einem abgeschlossenem System bleibt der Gesamtbetrag der Energie konstant.
 Innerhalb des Systems k\"onnenn sich die verschiedenen Energien ineinander umwandeln
 (1. Hauptsatz der Thermodynamik). Dabei sind die verschiedenen Energien die
 kinetische Energie $E_{kin}$, die potentielle Energie $E_{pot}$ und die innere Energie U
 des Systems. Diese innere Energie beschreibt die gesamte thermische Energie bzw.\ die
 gesamte kinetische und potentielle Energie des Systems. Die \"Anderung $dU$
 l\"asst sich mit der \"Anderung $dW$ der Arbeit und dem W\"arme\"ubertrag $dQ$
 bestimmen.
 \begin{align}
     \label{f1}
     dU=dQ+dW
 \end{align}
 Mithilfe des W\"arme\"ubertrags und der Temperatur\"anderung $dT$ l\"asst sich die
 W\"armekapazit\"at $C$ definieren:
 \begin{align}
     \label{f2}
     C=\frac{dQ}{dT}
 \end{align}
 Dies ist von der Masse $m$ ab\"angig, weswegen man die spezifische W\"armekapazit\"at
 \begin{align}
     \label{f3}
     c=\frac{C}{m}
 \end{align}
 definiert. M\"ochte man stattdessen eine Unabh\"angigkeit von der Stoffmenge M, so ist
 \begin{align}
     \label{f4}
     C_{m}=\frac{c}{n}
 \end{align}
 als molare W\"armekapazit\"at definiert, wobei n die Molanzahl ist. Die
 W\"armekapazit\"at ist stoffabh\"angig und wird in $\frac{Joule}{Kelvin}$ angegeben.
 \subsection{Gase}
 F\"ur ideale Gase (bestehend aus punktf\"ormigen isotropen Kugeln, welche nur
 elastische St\"o{\ss}e untereinander betreiben) gilt immer eine Zustandsgleichung.
 \begin{align}
     \label{f5}
     \frac{pV}{T}=konstant=n \cdot R=N \cdot k_{B}
 \end{align}
 Nach der kinetischen Gastheorie definiert man den Druck des Gases als Folge der
 St\"o{\ss}e, die die Gasmolek\"ule auf die Gef\"a{\ss}wand aus\"uben. \\
 Nun k\"onnen die Zust\"ande eines Gases verschieden ge\"andert werden. Als erstes sei
 die isochore Zustands\"anderung (v=konstant) genannt. F\"ur diese kann nun
 \begin{align}
     \label{f6}
     dU=C_{V}dT
 \end{align}
 \begin{align}
     \label{f7}
     C_{V}=N \frac{f}{2} k_{B}
 \end{align}
 hergeleitet werden. Die Freiheitsgrade sind die Anzahl der unabh\"angigen Parameter,
 die ben\"otigt werden, um ein System zu beschreiben. Es gibt 3 Freiheitsgrade der
 Translation, 2 der Rotation und 2 der Schwingung eines Molek\"uls bzw.\ Systems. \\
 Als zweites sei hier die isobare Zustands\"anderung eines Gases (p=konstant) genannt.
 Hierf\"ur wird \"ahnlich definiert:
 \begin{align}
     \label{f8}
     Vdp = C_{p}dT
 \end{align}
 und
 \begin{align}
     \label{f9}
     C_{p}=N\left(\frac{f}{2}+1\right)k_{B}
 \end{align}
 In einer dritten, sogenannten adiabatischen (isentropen) zustands\"anderung l\"asst
 sich das vorige zusammenbringen. Denn hierbei darf nur kein W\"armeaustausch mit der
 Umgebung stattfinden ($dQ=0$) bzw.\ die Entropie muss gleichbleibend sein.
 Demenstprechend finden auch nur reversible Prozesse statt. Damit wird aus~\eqref{f6}
 und~\eqref{f1}:
 \begin{align}
     \label{f10}
     dU=dW=-pdV=C_{V}dT
 \end{align}
 Mithilfe von~\eqref{f8} und~\eqref{f10}, indem man jeweils $dT$ isoliert, er\"alt man
 \begin{align}
     \label{f11}
     \frac{c_{p}}{c_{V}}Vdp=-pdV \leftrightarrow \frac{1}{p}dp=-\frac{\kappa}{V}dV
 \end{align}
 wobei
 \begin{align}
     \label{f12}
     \kappa = \frac{c_{p}}{c_{V}} 
 \end{align}
 benutzt wurde. Integrieren von $p_{1},V_{1} \rightarrow p_{2},V_{2}$ und exponenzieren
 von~\eqref{f11} ergibt
 \begin{align}
     \label{f13}
     \frac{p_{2}}{p_{1}} = {\left(\frac{V_{1}}{V_{2}}\right)}^{\kappa} \rightarrow
     p \cdot {V}^{\lappa} = konst.\
 \end{align}
 Mit~\eqref{f5} wird dies zu
 \begin{align}
     \label{f14}
     T\cdot{V}^{\kappa-1}=konst.\
 \end{align}
 und
 \begin{align}
     \label{f15}
     {p}^{1-\kappa}{T}^{\kappa}=konst.\
 \end{align}
 Die Gleichungen~\eqref{f13},\eqref{f14} und~\eqref{f15} werden Poisson-Gleichungen
 oder auch Adiabatengleichungen genannt. Der Isentropenindex $\kappa$ kann experimentell
 ermittelt werden. Dazu bieten sich mehrere Methoden an. Im Rahmen des Grundpraktikums
 werden die Methoden nach Clement-Desormes und nach Flammersfeld-R\"uchart verwendet. 
 \newpage
 \subsection{\underline{Methode nach Clement-Desormes}}
 Bei dieser Methode zur Bestimmung von $\kappa$ wird ein \"Uberdruck in einem
 Glasbeh\"alter erzeugt. Nach dem thermischem Ausgleich auf Zimmertemperatur wird ein
 Entl\"uftungsventil kurzzeitig ge\"offnet, wodurch eine nahezu afiabatische
 Zustands\"anderung erfolgt. Anschlie{\ss}end findet eine isochore Zustands\"anderung
 statt, bis die Zimmertemperatur wieder erreicht ist. \\
 Daher kann mit~\eqref{f13} und~\eqref{f14} f\"ur die adiabatische Zustands\"anderung
 folgendes geschrieben werden:
 \begin{align}
     \label{f16}
     \left(p_{0}+\Delta p_{1} \right) V_{0}^{\kappa}=p_{0}{\left(V_{0}+\Delta
     V\right)}^{\kappa}
 \end{align}
 \begin{align}
     \label{f17}
     \left(T_{0}-\Delta T\right){\left(V_{0}+\Delta V\right)}^{\kappa-1} = 
     T_{0}V_{0}^{\kappa-1}
 \end{align}
 mit $T_{0}$,$p_{0}$ und $V_{0}$ als Normalbedingungen (Zimmertemperatur, Luftdruck und 
 Volumen des Glasbeh\"alters) und $\Delta$p_{1}, $\Delta$V und $\Delta$T als \"Anderung
 dieser. Mit der N\"aherung $\Delta$V $<<$ $V_{0}$ erh\"alt man
 \begin{align}
     \label{f18}
 \end{align}
 \newpage
 \section{\underline{Tabelle 1: Methode nach Clement-Desormes:}}
 \begin{tabular}{FORMAT}
     Tabelleninhalt
 \end{tabular}
\end{document}
